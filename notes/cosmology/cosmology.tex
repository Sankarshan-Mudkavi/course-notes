\documentclass{article}

\usepackage{coursenotes}

\set{AuthorName}{TC Fraser}
\set{Email}{tcfraser@tcfraser.com}
\set{Website}{www.tcfraser.com}
\set{ClassName}{Cosmology}
\set{School}{University of Waterloo}
\set{CourseCode}{Phys 475}
\set{InstructorName}{Niayesh Afshrodi}
\set{Term}{Fall 2016}
\set{Version}{1.0}

\draftprofile[TC Fraser]{TC}{Purple}

\begin{document}

\titlePage

\tableOfContents

\disclaimer

\section{Introduction}


\subsection{History of Cosmology}

The first lecture consisted of everyone introducing themselves and then a brief summary of historical cosmology from Copernicus, to Kepler, Newton, and Einstein. The Copernican principle demonstrated that the earth is not special; Kepler's Laws revealed that the motion of the planets can be described by mathematical tools; Newton's laws unified physical properties observed on earth to those observed in the night sky. Finally, Einstein's equivalence principle further illuminated the equivalence between different observers. All of these observations and discovers have progressed us to the understanding we have today. The \textit{Cosmological Principle} is as follows,

\begin{center}
    \textit{At large scales the universe is homogeneous and isotropic. Equivalently, all observers see the same thing.}
\end{center}

However, there are two important caveats. First, the Cosmological Principle holds on very large scales (typically $\SI{6e22}{\m}$). Second, the Cosmological Principle holds for space but \textit{not} time. This latter caveat was not fully accepted until after Einstein. Einstein was under the motivation that the Universe was static and unchanging because of his unification of space and time (i.e. the homogeneity of space \textit{should} imply the homogeneity of time). However there was an observation that disagrees with this idea. \textbf{Olbers' Paradox} concerns itself with the issue of the darkness of the night sky. If the universe is homogeneous and isotropic, then in every direction one can look in the night sky, there should be a star at some distance away. In dual statement: no point in the night sky should be dark; hence the paradox. The resolution to Olbers' paradox is that the universe must not be infinite. \\

More rigorously, let the solid angle of an object a distance $r$ away with radius $R$ be $\pi R^2 / r^2$. Therefore the total solid angle for all stars should be,
\[ \sum_{i} \f{\pi R_{i}^2}{r_{i}^2} = n_* \intl_{0}^{r\tsb{max}} {4 \pi r^2 \dif r \f{\pi R_{*}^2}{r^2}} \propto r\tsb{max} R_{*}^2 \to \inf \]

In 1922, Hubble discovered the cosmic expansion of the universe which in turn implies the \textit{Big Bang}; following the ``linear'' expansion \textit{backward} in time, then at some point everything needs to be allocated at a singular point. \\

\textit{Remark:} In general, there does not seem to be a clear distinction between cosmology and astrophysics. For clarity, we will consider cosmology to be the evolution of the universe as a \textit{whole}. Of course there will be many exceptions to this focus, when we temporarily divert our attention to high energy particle physics, general relativity and other areas of physics. \\

\subsection{Studying the Universe as a Whole}

To study the paradigm of Cosmology, we will have to study the stuff that composes it. We can learn about the universe as a whole in many ways. For example, most of our observations are via the electromagnetic spectrum ($\ga$/X-ray, UV, optical, IR, $\mu$-waves, radio). Moreover we have the ability to probe the universe through neutrinos, cosmic rays and more recently due to the work of the LIGO observatory, gravitational waves. \\

The building blocks of the visible/optical universe are:
\begin{itemize}
    \item Stars
    \begin{itemize}
        \item Mass: $M \approx M_{\astrosun} \approx \SI{2e30}{\kg}$
        \item Distance: $D \gtrsim \SI{}{\pc} \approx \SI{3}{\lyr} \approx \SI{3e16}{\m} \approx \SI{2e5}{\AU} $
    \end{itemize}
    \item Galaxies
    \begin{itemize}
        \item Number of stars: $N \approx \SI{1e11}{}$
        \item Mass: $M \approx N \cdot M_{\astrosun}$
        \item Radius: $R \approx \SI{100}{\kilo\pc}$
    \end{itemize}
    \item Globular Clusters
\end{itemize}

\end{document}