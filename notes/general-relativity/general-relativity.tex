\documentclass{article}

\usepackage{coursenotes}

\set{AuthorName}{TC Fraser}
\set{Email}{tcfraser@tcfraser.com}
\set{Website}{www.tcfraser.com}
\set{ClassName}{General Relativity}
\set{School}{University of Waterloo}
\set{CourseCode}{Phys 476}
\set{InstructorName}{Florian Girelli}
\set{Term}{Winter 2016}

\begin{document}

\titlePage

\tableOfContents

\disclaimer


\section{Introduction}

\subsection{History}

The first lecture was a summary of astrophsyical history from around $\sim$200BC to today. I elected not to take notes as it was pretty standard stuff and a lot of slides. Sorry.

\section{Tensor Formalism}

At the core of General Relativity is the mathematics of differential geometry. Differential geometry requires the idea of tensors, a generalization of vectors and matricies and forms that can handle messy geometries and metrics. \\

Let $V$ be a vector space of finite dimension. Any $V$ is iosmorphic to $\R^{n+1}$ through the coefficients of a chosen basis. Let the basis of $V$ be given by,

\[ \bc{e_i}_{i=\tok0n} \]

Then any vector $v \in V$ is expressible by,

\[ v = \sum_{i=0}^{n} v^i e_i \]

Where $v^i$ are the $i$-th coefficients of the vector $v$ with respect to the basis $\bc{e_i}$.

\subsection{Einstein Summation Rule}

For convience let's provide a new, shorter notation for the vector $v$.

\[ v^ie_i = v^0e_0 + \ldots + v^ne_n = \sum_{i=0}^nv^ie_i \]

Effectively, we have just \textbf{dropped the summation sign}. The einstein summation rule is as follows: \\

If there are two identical indicies, 1 ``up'' and 1 ``down'', it means that a summation is secretly present, it's just be removed for convience. Note that the $i$ in this case is \textit{dummy index}.

\[ v^ie_i = v^{\alpha}e_{\alpha} = v^je_j \]

Here $v^i$ are the components of vector $v \in V$ and are real numbers. $v^i \in \R, \forall i \in \bc{\tok0n}$. \\

Note $v^i$ is called the vector $v$ when $i$ is the set $\bc{\tok0n}$, but can also be called the $i$-th component of $v$ when $i$ has a fixed value $i \in \bc{\tok0n}$. \\

\subsection{Examples of Basis for V}

The values of $e_i$ or the $i$'s themselves can take on many possible values.

\begin{itemize}
    \item cartesian coordinates $t,x,y,z$
    \item spherical coordinates $t, r, \phi, \theta$
    \item etc.
\end{itemize}

Each of the above examples is the space $V = \R^4$ (with some bounds for spherical coordinates).

\subsection{Dual Vector Space}

The dual vector space of $V$ denoted $V^*$ is also iosmorphic to $\R^{n+1}$ and is built from the space of linear forms on V.

\[ V^* = \bc{w: V \ar \R \st w(\alpha v_1 + \beta v_2) = \alpha w( v_1) + \beta w(v_2)} \]

where $v_1, v_2 \in V$ and $\alpha, \beta \in \R$.

In Quantum Mechanics, the vectors are the bras and the elements of the dual space (called the covectors) are the kets. \\

We note,

\[ \bc{f^i}_{i=\tok0n} \]

is the basis for $V^*$ is defined by the kronecker symbol $\delta$,

\[ f^j(e_j) = {\delta^j}_i \]

\[ {\delta^j}_i = \piecewise{1}{i=j}{0}{i\neq j} \]

An element in $V^*$ is $w = w_if^i$. $w_i$ are the components of the covector $w$. Note that for a \textbf{finite dimensional vector space},

\[ V^{**} = V \]

\subsection{Bilinear Maps}

Introduce a bilinear map $B(v, w)$ where $B: V \cross V \ar \R$ where,

\[ B(\alpha v_1 + \beta v_2, w) = \alpha B(v_1, w) + \beta B(v_2, w) \]

and the same for the other parameter $w$. \\

Examples include the inner product (otherwise known as the scale or dot product).

Bilinear forms are bilinear maps such that the following conditions are true:

\begin{itemize}
    \item symmetric: $B(v,w) = B(w,v)$
    \item non-degenerated: $B(v,w) = 0 \quad \forall v \implies w = 0$
\end{itemize}

Playing with indicies,

\begin{align*}
    B(v,w) &= B(v^\alpha e_\alpha, w^\beta e_\beta) \\
           &= v^\alpha B( e_\alpha, w^\beta e_\beta) \note{By linearity} \\
           &= v^\alpha w^\beta B( e_\alpha, e_\beta) \note{By linearity}
\end{align*}

A bilinear map used in this way provides a way to eliminate the headache of complicated cross sums. Define new notation,

\[ B( e_\alpha, e_\beta) \defined g_{\al\be} \]

Where $g_{\al\be}$ is a real number $\R$ whenever $\al$ and $\be$ are fixed.

\[ B(v, w) = v^\al w^\be g_{\al\be} = v^\al g_{\al\be}w^\be =  w^\be g_{\al\be} v^\al \]

All of the above terms are commutative because in the end, it represents a sum over all $\al, \be$.

\[ B(v,w) = \underbrace{v^0w^0g_{00} + \ldots + v^2w^3g_{2,3} + \ldots + v^nw^ng_{nn}}_{(n+1)^2 \text{terms}} \]

\subsection{Distance and Norms}

To define a distance in a vector space, we can use norms. In this case, $g_{\al\be}$ would be called the metric. The Euclidean metric (with respect to a cartesian basis) for example would be,

\[ g_{\al\be} = \piecewise{1}{\al = \be}{0}{\al \neq \be}  \]

We can also choose to enforce that the basis be orthonormal,

\[ B(e_i, e_j) = \piecewise{\pm1}{i = j}{0}{i \neq j} \]

Note that the potential for a negative norm means the notion of positive definiteness is no longer gauranteed.

\subsection{Signatures of Metrics}

We call the signature of the metric the number of $+1$'s and $-1$'s appearing in $g_{ij}$ when dealing with the orthonormal basis. Signature is denoted as:

\[ \br{p, q} = \br{\underbrace{p}_{\text{postive}}, \underbrace{q}_{\text{negative}}} \]

For example,

\begin{itemize}
    \item Euclidean metric: $(n+1, 0)$
    \item Minkowski metric: $(n, 1)$
\end{itemize}

Note the order of the signature is chosen to be $(p,q)$ and not $(q, p)$ by convention.

\subsection{Covectors from Vectors}

Note that $v^i$ was called the vector and $w_i$ was called the covector. This notation seems to indicate that conversion between $V$ and $V^*$ is notationally equivalent to raising and lowering the indicies. \\

We call the following opperation ``Lowing the index using the metric''.

\[ \underbrace{v^\al}_{\text{components of vector}} \mapsto g_{\al\be} v^\be = \underbrace{v_\al}_{\text{components of covector}} \]

In use,

\[ B(v,w) = v^\al g_{\al\be} w^\be = \untext{v_\be}{bra} \untext{w^\be}{ket} \]


\end{document}