\documentclass{article}

\usepackage{coursenotes}

\set{AuthorName}{TC Fraser}
\set{Email}{tcfraser@tcfraser.com}
\set{Website}{www.tcfraser.com}
\set{ClassName}{Statistical Mechanics}
\set{School}{University of Waterloo}
\set{CourseCode}{Phys 359}
\set{InstructorName}{Michel Gingras}
\set{Term}{Winter 2016}

\begin{document}

\titlePage

\tableOfContents

\disclaimer

\section{Introduction}

\subsection{What is Statistical Mechanics}

Statistical Mechanics is thearea of Physics interested in systems with a large number of degress of freedon $n$. Note that these variables can be interacting or not. \\

There are two distinct class of Statistical Mechanics: equilibrium and non-equilibrium. \\

The Statistical part of Statistical Mechanics implies that it is inherently a study of probabilities and probability distributions. These laws must still remain fully consistent with physical laws. \\

Typically, systems are analyzed on a microscopic level. For a system of particles $\bc{q_i}$ and their positions $\bc{\vr_i}$.

\end{document}