\documentclass{article}

\usepackage{coursenotes}

\set{AuthorName}{TC Fraser}
\set{Email}{tcfraser@tcfraser.com}
\set{Website}{www.tcfraser.com}
\set{ClassName}{Calculus of Variations}
\set{School}{University of Waterloo}
\set{CourseCode}{Amath 456}
\set{InstructorName}{David Siegel}
\set{Term}{Fall 2016}
\set{Version}{1.0}

\draftprofile[TC Fraser]{TC}{Purple}

\theoremstyle{plain}
\newtheorem{theorem}{Theorem}
\newtheorem{lemma}[theorem]{Lemma}
\newtheorem{corollary}[theorem]{Corollary}

\theoremstyle{definition}
\newtheorem{definition}[theorem]{Definition}
\newtheorem{example}[theorem]{Example}

\theoremstyle{remark}
\newtheorem{remark}[theorem]{Remark}

\begin{document}

\titlePage

\tableOfContents

\disclaimer

\section{Introduction}

As a first historical example, we will consider the \textbf{Brachistorchrone} first discussed by Johann Bernoulli in 1696. Consider to

\begin{center}
\begin{tikzpicture}
    \node (o) at (0,0) {$\br{0,0}$};
    \node (y) at (0,-4) {$y$};
    \node (x) at (4,0) {$x$};
    \node (a,b) at (3,-3) {$\br{a,b}$};
    \path[draw, ->] (o) -- (x);
    \path[draw, ->] (o) -- (y);
\end{tikzpicture}
\end{center}

\todo[TC]{Finish this thing}

Let $m$ be the mass of the bead. The conservation of energy is given by $1/2 m v^2$


\section{Minimization of Functionals}

\section{Extrema and Derivation of the Euler-Largrane Equations}

\section{Convexity}

\section{Extensions of The Basic Problem}

\section{Hamilton's Principle}

\section{Optimal Control}

\end{document}
